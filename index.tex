% Options for packages loaded elsewhere
\PassOptionsToPackage{unicode}{hyperref}
\PassOptionsToPackage{hyphens}{url}
\PassOptionsToPackage{dvipsnames,svgnames,x11names}{xcolor}
%
\documentclass[
  letterpaper,
  DIV=11,
  numbers=noendperiod]{scrreprt}

\usepackage{amsmath,amssymb}
\usepackage{iftex}
\ifPDFTeX
  \usepackage[T1]{fontenc}
  \usepackage[utf8]{inputenc}
  \usepackage{textcomp} % provide euro and other symbols
\else % if luatex or xetex
  \usepackage{unicode-math}
  \defaultfontfeatures{Scale=MatchLowercase}
  \defaultfontfeatures[\rmfamily]{Ligatures=TeX,Scale=1}
\fi
\usepackage{lmodern}
\ifPDFTeX\else  
    % xetex/luatex font selection
\fi
% Use upquote if available, for straight quotes in verbatim environments
\IfFileExists{upquote.sty}{\usepackage{upquote}}{}
\IfFileExists{microtype.sty}{% use microtype if available
  \usepackage[]{microtype}
  \UseMicrotypeSet[protrusion]{basicmath} % disable protrusion for tt fonts
}{}
\makeatletter
\@ifundefined{KOMAClassName}{% if non-KOMA class
  \IfFileExists{parskip.sty}{%
    \usepackage{parskip}
  }{% else
    \setlength{\parindent}{0pt}
    \setlength{\parskip}{6pt plus 2pt minus 1pt}}
}{% if KOMA class
  \KOMAoptions{parskip=half}}
\makeatother
\usepackage{xcolor}
\setlength{\emergencystretch}{3em} % prevent overfull lines
\setcounter{secnumdepth}{5}
% Make \paragraph and \subparagraph free-standing
\makeatletter
\ifx\paragraph\undefined\else
  \let\oldparagraph\paragraph
  \renewcommand{\paragraph}{
    \@ifstar
      \xxxParagraphStar
      \xxxParagraphNoStar
  }
  \newcommand{\xxxParagraphStar}[1]{\oldparagraph*{#1}\mbox{}}
  \newcommand{\xxxParagraphNoStar}[1]{\oldparagraph{#1}\mbox{}}
\fi
\ifx\subparagraph\undefined\else
  \let\oldsubparagraph\subparagraph
  \renewcommand{\subparagraph}{
    \@ifstar
      \xxxSubParagraphStar
      \xxxSubParagraphNoStar
  }
  \newcommand{\xxxSubParagraphStar}[1]{\oldsubparagraph*{#1}\mbox{}}
  \newcommand{\xxxSubParagraphNoStar}[1]{\oldsubparagraph{#1}\mbox{}}
\fi
\makeatother


\providecommand{\tightlist}{%
  \setlength{\itemsep}{0pt}\setlength{\parskip}{0pt}}\usepackage{longtable,booktabs,array}
\usepackage{calc} % for calculating minipage widths
% Correct order of tables after \paragraph or \subparagraph
\usepackage{etoolbox}
\makeatletter
\patchcmd\longtable{\par}{\if@noskipsec\mbox{}\fi\par}{}{}
\makeatother
% Allow footnotes in longtable head/foot
\IfFileExists{footnotehyper.sty}{\usepackage{footnotehyper}}{\usepackage{footnote}}
\makesavenoteenv{longtable}
\usepackage{graphicx}
\makeatletter
\def\maxwidth{\ifdim\Gin@nat@width>\linewidth\linewidth\else\Gin@nat@width\fi}
\def\maxheight{\ifdim\Gin@nat@height>\textheight\textheight\else\Gin@nat@height\fi}
\makeatother
% Scale images if necessary, so that they will not overflow the page
% margins by default, and it is still possible to overwrite the defaults
% using explicit options in \includegraphics[width, height, ...]{}
\setkeys{Gin}{width=\maxwidth,height=\maxheight,keepaspectratio}
% Set default figure placement to htbp
\makeatletter
\def\fps@figure{htbp}
\makeatother
% definitions for citeproc citations
\NewDocumentCommand\citeproctext{}{}
\NewDocumentCommand\citeproc{mm}{%
  \begingroup\def\citeproctext{#2}\cite{#1}\endgroup}
\makeatletter
 % allow citations to break across lines
 \let\@cite@ofmt\@firstofone
 % avoid brackets around text for \cite:
 \def\@biblabel#1{}
 \def\@cite#1#2{{#1\if@tempswa , #2\fi}}
\makeatother
\newlength{\cslhangindent}
\setlength{\cslhangindent}{1.5em}
\newlength{\csllabelwidth}
\setlength{\csllabelwidth}{3em}
\newenvironment{CSLReferences}[2] % #1 hanging-indent, #2 entry-spacing
 {\begin{list}{}{%
  \setlength{\itemindent}{0pt}
  \setlength{\leftmargin}{0pt}
  \setlength{\parsep}{0pt}
  % turn on hanging indent if param 1 is 1
  \ifodd #1
   \setlength{\leftmargin}{\cslhangindent}
   \setlength{\itemindent}{-1\cslhangindent}
  \fi
  % set entry spacing
  \setlength{\itemsep}{#2\baselineskip}}}
 {\end{list}}
\usepackage{calc}
\newcommand{\CSLBlock}[1]{\hfill\break\parbox[t]{\linewidth}{\strut\ignorespaces#1\strut}}
\newcommand{\CSLLeftMargin}[1]{\parbox[t]{\csllabelwidth}{\strut#1\strut}}
\newcommand{\CSLRightInline}[1]{\parbox[t]{\linewidth - \csllabelwidth}{\strut#1\strut}}
\newcommand{\CSLIndent}[1]{\hspace{\cslhangindent}#1}

\KOMAoption{captions}{tableheading}
\makeatletter
\@ifpackageloaded{bookmark}{}{\usepackage{bookmark}}
\makeatother
\makeatletter
\@ifpackageloaded{caption}{}{\usepackage{caption}}
\AtBeginDocument{%
\ifdefined\contentsname
  \renewcommand*\contentsname{Table of contents}
\else
  \newcommand\contentsname{Table of contents}
\fi
\ifdefined\listfigurename
  \renewcommand*\listfigurename{List of Figures}
\else
  \newcommand\listfigurename{List of Figures}
\fi
\ifdefined\listtablename
  \renewcommand*\listtablename{List of Tables}
\else
  \newcommand\listtablename{List of Tables}
\fi
\ifdefined\figurename
  \renewcommand*\figurename{Figure}
\else
  \newcommand\figurename{Figure}
\fi
\ifdefined\tablename
  \renewcommand*\tablename{Table}
\else
  \newcommand\tablename{Table}
\fi
}
\@ifpackageloaded{float}{}{\usepackage{float}}
\floatstyle{ruled}
\@ifundefined{c@chapter}{\newfloat{codelisting}{h}{lop}}{\newfloat{codelisting}{h}{lop}[chapter]}
\floatname{codelisting}{Listing}
\newcommand*\listoflistings{\listof{codelisting}{List of Listings}}
\makeatother
\makeatletter
\makeatother
\makeatletter
\@ifpackageloaded{caption}{}{\usepackage{caption}}
\@ifpackageloaded{subcaption}{}{\usepackage{subcaption}}
\makeatother

\ifLuaTeX
  \usepackage{selnolig}  % disable illegal ligatures
\fi
\usepackage{bookmark}

\IfFileExists{xurl.sty}{\usepackage{xurl}}{} % add URL line breaks if available
\urlstyle{same} % disable monospaced font for URLs
\hypersetup{
  pdftitle={Engineering Biology},
  pdfauthor={R. Clay Wright},
  colorlinks=true,
  linkcolor={blue},
  filecolor={Maroon},
  citecolor={Blue},
  urlcolor={Blue},
  pdfcreator={LaTeX via pandoc}}


\title{Engineering Biology}
\author{R. Clay Wright}
\date{2025-05-02}

\begin{document}
\maketitle

\renewcommand*\contentsname{Table of contents}
{
\hypersetup{linkcolor=}
\setcounter{tocdepth}{2}
\tableofcontents
}

\bookmarksetup{startatroot}

\chapter*{Preface}\label{preface}
\addcontentsline{toc}{chapter}{Preface}

\markboth{Preface}{Preface}

This book supports the Synthetic Biology CuRE course at Virginia Tech
BSE 4014/5014.

\section*{Acknowledgments}\label{acknowledgments}
\addcontentsline{toc}{section}{Acknowledgments}

\markright{Acknowledgments}

The development of this course and the associated text here has been
supported by many people and funding organizations to which I am
extremely grateful.

Thanks to my colleagues at Virginia Tech who encouraged me to develop
this course. Thanks to my mentors Jennifer Nemhauser, Eric Klavins, and
Britney Moss for fostering my interest and training in pedagogy and
helping me with the scaffold for the initial iteration of this course.
Thanks to all of the students who have given me helpful feedback about
the course and materials along the way.

Thanks to the Virginia Space Grant Consortium for providing funding to
develop the first iteration of this course. Thanks to Opentrons for
providing an OT-2 for the course and discounts on pipettes and modules.
Thanks to the VT Office of Undergraduate Research for providing support
for developing scalable projects to summer undergraduate researchers.

This is a Quarto book.

To learn more about Quarto books visit
\url{https://quarto.org/docs/books}.

\bookmarksetup{startatroot}

\chapter{Syllabus}\label{syllabus}

\bookmarksetup{startatroot}

\chapter{Kit of Parts}\label{kit-of-parts}

\bookmarksetup{startatroot}

\chapter{Review of
Biology/Biochemistry}\label{review-of-biologybiochemistry}

\bookmarksetup{startatroot}

\chapter{Recombinant DNA technology}\label{recombinant-dna-technology}

\bookmarksetup{startatroot}

\chapter{Introduction to Auxin}\label{introduction-to-auxin}

\bookmarksetup{startatroot}

\chapter{Lab orientation: Safety, Pippetting, and Aseptic
technique}\label{lab-orientation-safety-pippetting-and-aseptic-technique}

\bookmarksetup{startatroot}

\chapter{Biological Parts (Protein expression and
plasmids)}\label{biological-parts-protein-expression-and-plasmids}

\bookmarksetup{startatroot}

\chapter{Bioinformatics: Tools for working with biological
sequences}\label{bioinformatics-tools-for-working-with-biological-sequences}

\bookmarksetup{startatroot}

\chapter{Modular Cloning}\label{modular-cloning}

\section{Introduction}\label{introduction}

This chapter introduces the principles of \textbf{Modular Cloning
(MoClo)}, a standardized method for assembling genetic circuits using
reusable DNA parts. MoClo is central to synthetic biology and enables
rapid, reliable construction of multigene constructs for functional
analysis and synthetic circuit design.

\section{What is MoClo?}\label{what-is-moclo}

MoClo, or Modular Cloning, is a Golden Gate-based cloning strategy that
uses Type IIS restriction enzymes to assemble DNA fragments with
predefined overhangs in a single reaction.

\subsection{Key Features}\label{key-features}

\begin{itemize}
\tightlist
\item
  \textbf{Type IIS restriction enzymes} like \textbf{BsaI} and
  \textbf{BsmBI} cut outside their recognition sites, enabling scarless
  assembly.
\item
  Each DNA part (e.g., promoter, coding sequence, terminator) is cloned
  into a \textbf{Level 0 plasmid} with specific flanking sequences.
\item
  Level 0 parts are assembled into \textbf{Level 1 cassettes}, which can
  then be assembled into \textbf{Level 2 multigene constructs}.
\end{itemize}

\begin{quote}
TODO: Insert diagram of MoClo hierarchy (L0 → L1 → L2).
\end{quote}

\section{Why Use MoClo?}\label{why-use-moclo}

\begin{itemize}
\tightlist
\item
  \textbf{Standardization}: Allows parts to be easily shared, reused,
  and recombined.
\item
  \textbf{Speed}: One-pot reactions reduce cloning time.
\item
  \textbf{Efficiency}: High-fidelity assembly with minimal hands-on
  steps.
\item
  \textbf{Flexibility}: Easily test multiple combinations of parts
  (e.g., promoter variants).
\end{itemize}

\section{Part Plasmids and Assembly
Strategy}\label{part-plasmids-and-assembly-strategy}

Each genetic element is cloned into its own plasmid, called a
\textbf{part plasmid}. These are stored in \textbf{E. coli} for easy
amplification and reuse.

\subsection{Advantages of Using Plasmid
Libraries}\label{advantages-of-using-plasmid-libraries}

\begin{itemize}
\tightlist
\item
  Easy to grow and miniprep.
\item
  Stable storage and replication.
\item
  Eliminates need to PCR amplify parts each time.
\end{itemize}

\begin{quote}
TODO: Add table of standard part types and corresponding overhangs.
\end{quote}

\section{MoClo Assembly Overview}\label{moclo-assembly-overview}

\begin{longtable}[]{@{}
  >{\raggedright\arraybackslash}p{(\columnwidth - 2\tabcolsep) * \real{0.3158}}
  >{\raggedright\arraybackslash}p{(\columnwidth - 2\tabcolsep) * \real{0.6842}}@{}}
\toprule\noalign{}
\begin{minipage}[b]{\linewidth}\raggedright
Step
\end{minipage} & \begin{minipage}[b]{\linewidth}\raggedright
Description
\end{minipage} \\
\midrule\noalign{}
\endhead
\bottomrule\noalign{}
\endlastfoot
1 & Clone each part into a Level 0 vector with correct overhangs. \\
2 & Use \textbf{BsaI} to assemble Level 0 parts into a Level 1
cassette. \\
3 & Use \textbf{BsmBI} to assemble multiple Level 1 cassettes into a
Level 2 construct. \\
\end{longtable}

Junctions between parts are defined by 4 bp overhangs, each unique to a
position in the expression cassette (e.g., promoter--CDS,
CDS--terminator).

\begin{quote}
TODO: Include schematic of overhang design and part compatibility.
\end{quote}

\section{Experimental Design Using
MoClo}\label{experimental-design-using-moclo}

Students used the MoClo system to design constructs testing different
auxin-responsive promoters and ARF (Auxin Response Factor) activators.

\subsection{Example Experimental
Goals}\label{example-experimental-goals}

\begin{itemize}
\tightlist
\item
  Compare activity of full-length ARF6 with truncated variants.
\item
  Test reporter expression from different auxin-responsive promoters.
\item
  Include negative controls using mutant promoters lacking ARF binding
  sites.
\end{itemize}

\subsection{Promoters and Controls}\label{promoters-and-controls}

\begin{longtable}[]{@{}
  >{\raggedright\arraybackslash}p{(\columnwidth - 2\tabcolsep) * \real{0.4348}}
  >{\raggedright\arraybackslash}p{(\columnwidth - 2\tabcolsep) * \real{0.5652}}@{}}
\toprule\noalign{}
\begin{minipage}[b]{\linewidth}\raggedright
Promoter
\end{minipage} & \begin{minipage}[b]{\linewidth}\raggedright
Description
\end{minipage} \\
\midrule\noalign{}
\endhead
\bottomrule\noalign{}
\endlastfoot
PIA19 & Auxin-responsive promoter \\
PIA19mut & Mutated version, lacks ARF binding sites (negative
control) \\
PER7 & Alternative auxin-responsive promoter \\
P32x & Promoter with moderate activation potential \\
\end{longtable}

\section{Best Practices in MoClo}\label{best-practices-in-moclo}

\begin{itemize}
\tightlist
\item
  Avoid reuse of identical sequences (e.g., same terminator) across
  constructs to reduce recombination.
\item
  Use \textbf{unique promoters and terminators} in multigene constructs.
\item
  Maintain \textbf{codon frame} at junctions (especially for CDS
  fusions).
\item
  Add \textbf{GG} or \textbf{stop codons} where appropriate.
\end{itemize}

\section{Advanced Tools and
Resources}\label{advanced-tools-and-resources}

\subsection{Toolkits}\label{toolkits}

\begin{itemize}
\tightlist
\item
  \textbf{Yeast Toolkit (YTK)}: Includes a wide variety of parts for use
  in \emph{S. cerevisiae}.
\item
  \textbf{Auxin Toolkit}: Custom parts developed for studying auxin
  signaling.
\end{itemize}

\subsection{Hardware}\label{hardware}

\begin{itemize}
\tightlist
\item
  Protocols for \textbf{robotic minipreps} are being developed to
  automate DNA extraction from plasmid libraries stored in 96-well
  plates.
\end{itemize}

\begin{quote}
TODO: Link to available toolkit maps and protocols.
\end{quote}

\section{Application: Combinatorial
Design}\label{application-combinatorial-design}

Students designed and tested modular constructs combining:

\begin{itemize}
\tightlist
\item
  Reporters (e.g., \textbf{mRuby2} with \textbf{UBM} degradation tags)
\item
  Activators (e.g., \textbf{ARF19}, \textbf{ARF6} variants)
\item
  Responsive promoters (e.g., \textbf{PIN18}, \textbf{PER7})
\end{itemize}

This modular approach allowed rapid testing of hypotheses such as:

\begin{itemize}
\tightlist
\item
  Optimal ARF expression levels for activation
\item
  Promoter responsiveness to auxin
\item
  Reporter signal tuning via degradation
\end{itemize}

\section{Summary}\label{summary}

MoClo enables students to move from hypothesis to construct in a
streamlined and standardized way. By combining software design, part
libraries, and efficient cloning, students can explore complex
biological questions with scalable, reproducible tools.

\begin{quote}
TODO: Add student planning worksheet for construct design.
\end{quote}

\section{Next Steps}\label{next-steps}

\begin{itemize}
\tightlist
\item
  Begin primer design and part selection for Level 0 construction.
\item
  Simulate and test assemblies using Benchling or Ape.
\item
  Prepare to begin Golden Gate assembly in lab next week.
\end{itemize}

\bookmarksetup{startatroot}

\chapter{Hypotheses, Research Questions, and Experimental
Design}\label{hypotheses-research-questions-and-experimental-design}

\bookmarksetup{startatroot}

\chapter{Level 0 Part Plasmid Design}\label{level-0-part-plasmid-design}

\section{Overview}\label{overview}

This chapter introduces students to experimental design in the context
of synthetic biology and the practical construction of Level 0 part
plasmids using Modular Cloning (MoClo) and the Yeast Toolkit (YTK).

We build on the previous lecture's discussion of MoClo principles and
emphasize hands-on experience with designing and assembling modular DNA
parts for use in downstream genetic circuits.

\section{Review of Modular Cloning
(MoClo)}\label{review-of-modular-cloning-moclo}

Modular Cloning allows the assembly of genetic parts---promoters, coding
sequences, terminators, and other regulatory elements---into
standardized expression cassettes. Each part is cloned into a
\textbf{Level 0 plasmid}, which can be stored, replicated, and reused.

\subsection{Key Principles}\label{key-principles}

\begin{itemize}
\tightlist
\item
  Uses \textbf{Type IIS restriction enzymes} (e.g., BsaI and BsmBI) that
  cut outside their recognition sites, allowing for \textbf{scarless
  assembly}.
\item
  Each part is flanked by \textbf{standardized 4 bp overhangs}, which
  determine its position and compatibility within an expression
  cassette.
\item
  Level 0 parts can be assembled into \textbf{Level 1 cassettes} (single
  gene), and multiple Level 1 cassettes can be assembled into
  \textbf{Level 2 constructs} (multigene systems).
\end{itemize}

\begin{quote}
TODO: Insert diagram illustrating MoClo hierarchy (L0, L1, L2).
\end{quote}

\section{Software Tools for Primer and Part
Design}\label{software-tools-for-primer-and-part-design}

\subsection{Benchling}\label{benchling}

Benchling is a browser-based molecular biology tool for sequence
annotation, primer design, and cloning simulations.

\textbf{To use Benchling:}

\begin{enumerate}
\def\labelenumi{\arabic{enumi}.}
\tightlist
\item
  Import or create your DNA sequence.
\item
  Annotate regions (promoter, CDS, terminator).
\item
  Use the \emph{``Primer''} tool to design and label forward and reverse
  primers.
\item
  For reverse primers, use the \emph{``Copy Special \textgreater{}
  Reverse Complement''} option.
\item
  Simulate Golden Gate assembly using \emph{Assembly Wizards} or manual
  cloning.
\end{enumerate}

\subsection{Ape (A Plasmid Editor)}\label{ape-a-plasmid-editor}

Ape is a downloadable tool that allows simple manipulation of DNA
sequences.

\begin{itemize}
\tightlist
\item
  Supports reverse complement operations and editing.
\item
  Useful for flipping 3′ to 5′ primers back into 5′ to 3′ notation.
\end{itemize}

\begin{quote}
TODO: Include screenshot of Benchling or Ape with primer annotations.
\end{quote}

\section{Activity Instructions}\label{activity-instructions}

Students worked from:

\begin{itemize}
\tightlist
\item
  \textbf{YTK Assembly Manual (pages 12--26)}: This guide details how to
  construct each part type using standardized overhangs and restriction
  sites.
\item
  \textbf{YTK Overhangs Primers Sheet (Google Sheets)}: Shared class
  document for assigning and recording primer sequences and overhangs.
\end{itemize}

\subsection{Group Workflow}\label{group-workflow}

\begin{enumerate}
\def\labelenumi{\arabic{enumi}.}
\tightlist
\item
  Open the manual and locate your assigned part type (e.g., Type 1, 2,
  3A, 3B\ldots).
\item
  Record the \textbf{part name}, \textbf{5′ overhang}, \textbf{3′
  overhang}.
\item
  Design forward and reverse primers:

  \begin{itemize}
  \tightlist
  \item
    Forward primer = {[}GC clamp{]} + {[}BsaI site{]} + {[}part-specific
    overhang{]} + {[}homology to template{]}
  \item
    Reverse primer = Same logic, but \textbf{reverse complement} of the
    sequence
  \end{itemize}
\end{enumerate}

\begin{quote}
TODO: Add primer design template for students to follow.
\end{quote}

\subsection{Primer Design Guidelines}\label{primer-design-guidelines}

\begin{itemize}
\tightlist
\item
  Primers should always be written 5′ → 3′.
\item
  Include at least 4 bp \textbf{GC clamp} before the restriction site.
\item
  Use \textbf{BsaI} recognition site (GGTCTC) for Level 0 assembly.
\item
  Add part-specific overhangs immediately downstream of the enzyme site.
\item
  Add \textbf{GG} for fusion proteins (e.g., Type 3B parts) to maintain
  codon frame.
\item
  Include a \textbf{stop codon} and/or \textbf{XhoI site} for terminator
  parts as needed.
\end{itemize}

\begin{quote}
TODO: Insert example primer sequence and breakdown of its components.
\end{quote}

\section{Practical Considerations}\label{practical-considerations}

\subsection{In-Frame Assembly}\label{in-frame-assembly}

\begin{itemize}
\tightlist
\item
  Coding sequences often require special care to preserve the reading
  frame across part boundaries.
\item
  For example, combining Type 3A and 3B parts (both coding) needs a GG
  insertion to keep translation in-frame.
\end{itemize}

\subsection{Negative Controls}\label{negative-controls}

\begin{itemize}
\tightlist
\item
  Use mutant promoters (e.g., PIA19mut) that lack binding sites to
  demonstrate ARF-specific activation.
\item
  Design primers that omit ARFs or use constitutively inactive variants
  as negative controls.
\end{itemize}

\begin{quote}
TODO: Provide table of part types and expected features (start codon,
stop codon, etc.).
\end{quote}

\section{Constructing Level 0 Parts}\label{constructing-level-0-parts}

To make a new Level 0 part:

\begin{enumerate}
\def\labelenumi{\arabic{enumi}.}
\tightlist
\item
  Identify the \textbf{source sequence} (synthesized, genomic DNA,
  etc.).
\item
  Design primers to amplify the sequence:

  \begin{itemize}
  \tightlist
  \item
    Add \textbf{BsaI site}, overhangs, and homology arms.
  \end{itemize}
\item
  PCR amplify the part.
\item
  Digest both the PCR product and \textbf{Level 0 entry vector} with
  BsaI.
\item
  Ligate the digested insert and vector.
\item
  Transform into competent \textbf{E. coli} and select colonies.
\item
  Screen clones by colony PCR or sequencing.
\end{enumerate}

\begin{quote}
TODO: Include diagram showing Level 0 assembly via Golden Gate.
\end{quote}

\section{Troubleshooting Tips}\label{troubleshooting-tips}

\begin{itemize}
\tightlist
\item
  Double-check primer orientation and melting temperatures (55--72°C).
\item
  Avoid internal BsaI or BsmBI sites within the part sequence.
\item
  Use Benchling to simulate and verify assembly.
\end{itemize}

\section{Summary}\label{summary-1}

This activity combined theoretical and practical training in modular
cloning. Students learned to:

\begin{itemize}
\tightlist
\item
  Navigate cloning toolkits and design compatible parts
\item
  Use software to annotate sequences and design primers
\item
  Understand the logic of MoClo part assembly
\end{itemize}

These skills will be applied to build expression cassettes and multigene
constructs for synthetic biology experiments in upcoming weeks.

\begin{quote}
TODO: Link to shared Google Sheet with part assignments and example
sequences.
\end{quote}

\section{Next Steps}\label{next-steps-1}

\begin{itemize}
\tightlist
\item
  Finalize primer designs for assigned part types.
\item
  Submit sequences for synthesis or begin PCR amplification.
\item
  Prepare for Level 0 part cloning and validation.
\end{itemize}

\bookmarksetup{startatroot}

\chapter{Measuring Biology}\label{measuring-biology}

\bookmarksetup{startatroot}

\chapter{Level 1 Transcriptional Unit
Design}\label{level-1-transcriptional-unit-design}

\bookmarksetup{startatroot}

\chapter{Level 2 Multicassette Plasmid
Design}\label{level-2-multicassette-plasmid-design}

\bookmarksetup{startatroot}

\chapter{Finalizing Designs}\label{finalizing-designs}

\bookmarksetup{startatroot}

\chapter{Opentrons Lab automation}\label{opentrons-lab-automation}

\bookmarksetup{startatroot}

\chapter{Minipreps}\label{minipreps}

\bookmarksetup{startatroot}

\chapter{DNA spectrophotometry}\label{dna-spectrophotometry}

\bookmarksetup{startatroot}

\chapter{DNA Assembly Methods}\label{dna-assembly-methods}

\bookmarksetup{startatroot}

\chapter{Setting up Level 1
Assemblies}\label{setting-up-level-1-assemblies}

\bookmarksetup{startatroot}

\chapter{Final projects}\label{final-projects}

\bookmarksetup{startatroot}

\chapter{PCR primer design}\label{pcr-primer-design}

\bookmarksetup{startatroot}

\chapter{Gel Electrophoresis}\label{gel-electrophoresis}

\bookmarksetup{startatroot}

\chapter{Colony PCR}\label{colony-pcr}

\bookmarksetup{startatroot}

\chapter{Genome Engineering}\label{genome-engineering}

\bookmarksetup{startatroot}

\chapter{Yeast Transformation}\label{yeast-transformation}

\bookmarksetup{startatroot}

\chapter{Sequencing Analysis}\label{sequencing-analysis}

\bookmarksetup{startatroot}

\chapter{Flow Cytometry}\label{flow-cytometry}

\bookmarksetup{startatroot}

\chapter*{References}\label{references}
\addcontentsline{toc}{chapter}{References}

\markboth{References}{References}

\phantomsection\label{refs}
\begin{CSLReferences}{0}{1}
\end{CSLReferences}




\end{document}
